
\subsection{Introducing Cfruby, Cfyaml and libcfruby}

\emph{Powerful next generation tools for managing large scale heterogeneous
computing environments, by David Powers and Pjotr Prins}

\setcounter{section}{1}

\subsubsection{Overview}

The \cfruby\ project is an open source software initiative with
the dual purpose of (1) providing a solid library for system
administration and (2) tools for emulating and superseding the workings of
GNU \cfengine.

Large scale networked computing environments need appropriate tools
for managing system and computational services. Cfruby is a tool-set
that allows full automation, transactions, granular locking, versioned
backups, extensive and configurable logging, a state machine with
\cfengine\ style classes and scripting.

The project has two types of users. The first type is the system
administrator who wants to replace \cfengine\ with something more
flexible and advanced. The second type is the Ruby developer, who
wants to use powerful system administration features from his
software. Since both worlds are served it means a broader user base
which adds to the robustness and reliability of the libraries. 

Here some concepts are introduced so you can get an appreciation
for what can be done.


\subsubsection{\cfenjin, a scripting language and state engine}

\cfenjin\ is a parser and part of the \cfruby\ tool-set. It is made to
look and behave like \cfengine, the GNU configuration engine, which was
conceived in 1994 by Mark Burgess\cite{Cfengine} and has a simple
programming language that can be mastered by any qualified system
administrator. 

\vspace{0.2in}

\vbox{
\cfengine\ was superior to other tools available at the time because it allowed a level
of automating system administration while pursuing:

\begin{itemize}
  \item Readability
  \item Documentation
  \item Prevention of duplication
  \item Abstraction
\end{itemize}
}

\vbox{
\cfenjin\ aims to be a better \cfengine. Like \cfengine\ it

\begin{itemize}
  \item is a \emph{simple} language
  \item has a class based decision structure, which documents policies
  \item allows automation and abstraction
\end{itemize}
}

But in addition it

\begin{itemize}
  \item allows embedded \ruby
  \item can be relatively easily extended
  \item has extensive and configurable logging support
\end{itemize}

Instead of opting for the original \cfengine\ code base we decided to
use the \ruby\ language as a natural parser.

Ruby is a fully object oriented scripting language\cite{Ruby} with a
rapidly growing support base. \ruby\ lends itself very well to rapid
programming and usually results in terse but readable
code\cite{Prins:Ruby2002}. It's affiliation to Perl is close enough for most
Perl programmers to quickly pick up Ruby. Furthermore \ruby\ is
nowadays on most installed \unix\ systems and runs on \mswindows\ as well.

By using \ruby\ as a base language we get a very powerful scripting
language for free.  Besides being well tested, cross-platform and secure, it has 
a number of nice features out of the box (support for Perl-like regular 
expressions, arrays, hashes, a great syntax checker, libraries for XML parsing, FTP and
LDAP support, YAML etc. thrown in) that have been listed
on \cfengine's wish list for a long time.

Meanwhile \cfenjin\ users do \emph{not} need to learn \ruby\ as
\cfenjin\ is a self-contained language implemented as a pre-parser (the
output is pure \ruby\ code).


\vbox{
\subsubsection*{Example 1: Classes \& configuration}

\cfenjin\ has classes and logical execution.

\scriptsize
\begin{verbatim}

  servers = ( screamer flightless )
  sshd    = ( servers )

sshd::

  $sshd_groups = 'sshlogin'

\end{verbatim}
\normalsize

Here we tell \cfenjin\ that the servers are running sshd. If sshd is
defined \$sshd\_groups gets set\footnote{Ruby aficionados should note
  that \cfenjin\ handles variable scope in a way that localizes global
  names to the local instance and class names to the package}.
}

Next we can use logic like

\scriptsize
\begin{verbatim}
package 'ssh','openssh'

control:

    if File.directory? '/etc/ssh'
      @ssh_etc = '/etc/ssh'
    else
      @ssh_etc = '/etc'
    end
   
files:

    @ssh_etc+'/ssh_config' o=root m=0644
    @ssh_etc+'/sshd_config' o=root m=0600
    @ssh_etc+'/ssh_host_dsa_key' o=root m=0400
    @ssh_etc+'/ssh_host_rsa_key' o=root m=0400
    if File.exist? @ssh_etc+'/ssh_host_key'
      @ssh_etc+'/ssh_host_key' o=root m=0400
    end

editfiles:

    EditFile.edit @ssh_etc+'/sshd_config' do | f |
      f.ReplaceAllAppend "UsePrivilegeSeparation [Nn].*",
                         "UsePrivilegeSeparation yes"
      f.ReplaceAllAppend "PermitRootLogin\s+[yY].*",
                         "PermitRootLogin no"
      f.ReplaceAllAppend "PermitEmptyPasswords[[:space:]]+[yY].*",
                         "PermitEmptyPasswords no"
      f.ReplaceAllAppend "^AllowGroups[[:space:]]+.*",
                         "AllowGroups #{$sshd_groups}" if $sshd_groups
      f.ReplaceAllAppend "^AllowUsers[[:space:]]+.*",
                         "AllowUsers #{$sshd_users}" if $sshd_users
    end

\end{verbatim}
\normalsize

where we test where the configuration files are (note we are using
plain \ruby\ in the control block). In the files block we check and
set permissions and in the 'editfiles' block we edit the configuration
file.

Note that editfiles allows for GNU type regular expressions, as well
as \ruby\ and Perl type. Also the 'editfiles' block allows for 'if'
blocks and other \ruby\ language syntax.

\subsubsection*{Example 2: Package}

\cfenjin\ understands the package system. By using the 'package' command
in a script it will skip configuration of a non-installed package. E.g.

\begin{verbatim}
  package 'sudo'

  ... do 'sudo' stuff when package 'sudo' installed ...
\end{verbatim}

this is why in Example 1 we do not need to surround every statement
with an 'sshd::' clause(!)

\subsubsection*{Example 3: Tidy}

\cfenjin\ has powerful tidying. The pattern search allows for full
regular expressions. For example, to tidy all files older than 30 days
matching:

\begin{verbatim}

  tidy:

    /var/log/ pattern='mail\.\d+\.gz' age=30
\end{verbatim}

removes /var/log/mail.3.gz when older than 30 days.

\subsubsection{\cfyaml, another state engine}

If you want to use a code generator, or parse configuration files,
a YAML format can be used instead (or you could implement an \xml\ edition
quite easily. For example the following may be a way of configuring
your mailer installation:

\begin{verbatim}
- install: exim
- filename: /usr/local/etc/exim
  options:
    recursive: true
  user: mailnull
  group: mail
  mode: u=rwX,g=rX,o-rwx
- filename: /etc/rc.conf
  set:
    - [sendmail_enable, '"NONE"']
    - [exim_enable, '"YES"']
    - [exim_flags, '"-bd -q5m"']
\end{verbatim}

\subsubsection{\libcfruby, the library}

By studying the rdoc generated help files you can get a taste for the
number of classes and methods that are available to Ruby programmers.

Some quick examples:

\scriptsize
\begin{verbatim}
   # user and group management
   os = Cfruby::OS::OSFactory.new().get_os()
   @manager = os.get_user_manager() 
   grouplist = @manager.groups()
   assert_equal(true, @manager.group?('bin'))
   @manager.add_user("lijygrdwa")

   # copying
   Cfruby::FileOps.backup(@basefile.path)
   Cfruby::FileOps.copy(copydirpath, newfilepath)

   # tidying
   Cfruby::FileOps.delete(@basedir, :glob => "^mail\.\d+\.log\.gz", 
     :recursive => true)

\end{verbatim}
\normalsize

and, unlike standard \ruby\ library calls you get full and
configurable logging. An extensive version of the log can look like:

\scriptsize
\begin{verbatim}
cfruby info [2] copy /mnt/flash/darcs/cf/cfwrk/masterfiles/Xresources to 
  /home/wrk/.Xresources - attempt
cfruby info [3] copy /mnt/flash/darcs/cf/cfwrk/masterfiles/Xresources to 
  /home/wrk/.Xresources - aborted - files have the same sha1 hash
cfruby info [2] changing permissions of "/home/wrk/.Xresources" to "400" - attempt
cfruby info [1] changing permissions of "/home/wrk/.Xresources" to "400" - done
cfruby info [1] Computer "palermo" belongs to classes "any, debianlinux, 
  laptop, linux, pacs, wrk"
\end{verbatim}
\normalsize

And it is easy to tune it down to any level you want. For example you
can choose to only log changes really made to the system. You can
disable/enable parts of the logger and control where to send the
output.

\subsubsection{Where are we now?}

The documentation still has some holes, and needs further elaboration
and work. Nevertheless \cfruby\ as well as the library is rather
complete, can be considered beta, and is being used in a number of
production systems. Late 2005 we will export a stable 1.0 edition of
\cfruby. 

Examples of \cfengine\ style scripts can be found in the
./documentation/cfenjin/examples directory. So if you are armed with
the standard GNU \cfengine\ documentation and a \ruby\ manual you
should already be able to use \cfruby\ to your advantage.

\subsubsection{Availability}

\cfruby\ is open source and licensed using the GPL. Sources and
documentation can be obtained from

\begin{verbatim}
  http://rubyforge.org/projects/cfruby/
\end{verbatim}







