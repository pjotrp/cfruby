
\subsection{Logging and Debugging in Cfruby}

Cfruby serves a dual purpose, first as a Ruby library for system
administration (cfrubylib) and second as a replacement for the
Cfengine tool (Cfruby), we realised we had conflicting requirements
handling output messages at the library level.

The library 'cfrubylib' is to be used from (generic) Ruby programs and
any output messages should be controlled by the calling program - and
often there should be no output whatsoever.

On the other hand Cfruby - the Cfengine like language and parser -
requires logging and verbosity at the library level - otherwise all
the logging 'logic' has to be handled before and after calling into
library functions. There is duplication of code if one handles it like
this:

\begin{verbatim}
log 'Message1' if !File.exist? fn1
log 'Message2' if File.exist? fn2
if !lib.cp fn1,fn2
  log 'Message3' 
end
(...)
\end{verbatim}

You can imagine the File.exist? statements also happen inside the
method 'lib.cp'. 

At this stage we came up with a logger class that can be overridden by
a program, or by Cfruby itself. The standard library calls into an
empty default. Simplistically you would create your own logger and
override the standard version:

\begin{verbatim}
class Logger

  def message m
    'do something'
  end

end
\end{verbatim}

Next we realised that we also do not want to clutter the library code
itself with logging messages(!) At this stage the library code would
look like this:

\begin{verbatim}
def cp fn1, fn2
  logger.message 'Message1' if !File.exist? fn1
  logger.message 'Message2' if File.exist? fn2
  if !Fileutils.cp fn1,fn2
    logger.message 'Message3' 
  else
    logger.message 'Message4'
  end
end
(...)
\end{verbatim}

You can see it is still ugly and takes attention away from the main
code. Next we thought we should use closures to reduce the number
of messages:

\begin{verbatim}
def cp fn1, fn2
  logger.test('Message',fn1) {
    File.exist? fn1
  }
  logger.test('Message',fn2) {
    File.exist? fn2
  }
  logger.act('Message',fn1,fn2) {
    Fileutils.cp fn1,fn2
  }
end
\end{verbatim}

Where failure is handled through exceptions - so the logger.act method
'knows' which messages to display. This setup has the additional
benefit that the implementation of 'logger.act' can also switch of
execution of the code(!) I.e. when running the program in --dry-run
mode. Nice!

Finally the masterstroke - Dave Powers turned the logger into a more
generic 'flowmonitor' which leads to code like

\begin{verbatim}
    # Remove a directory entry. +dirname+ can be a list.
    def FileOperations.rmdir(dirname)
      if(dirname.kind_of?(String))
        dirname = Array.[](dirname)
      end

      dirname.each do | d |
        Cfruby.controller.attempt("rmdir #{d}", 'nonreversible', 'destructive')
        {
          FileUtils.rmdir d
          Cfruby.controller.inform('debug', "deleted directory \"#{d}\"")
        }
      end
    end
\end{verbatim}

The idea is that we use something like a modified observer, where the
library simple tells the observer for each logical action (which may
be a largish chunk of real code) what it intends to do, and also some
general (probably arbitrary tag) information about the type of action
that it is.  The code calling the library registers as an observer and
it informed about each action, at which point it can do whatever it
wants with that information (log it, store it, discard it, etc) and
can *respond* to the message - indicating that it should or should not
carry out the action.  Then it is in-turn informed again with a
success (and maybe failure?) message.  The point here is that this
sort of thing is much more suitable for a library that is meant to be
driven and used in a potentially fine-grained manner.  The logging can
now be modified in the parser and driving code to your hearts content.
Only want to log some types of actions, fine.  Only want summary
output, fine.  Want it all, fine.  Want a dry run, fine.  Only want
actions that don't touch the network, fine... so on and so on.  This
is really only limited by the quality of the message we pass and how
finely grained we want to make it in the library code.

If you attach an Observer to the Cfruby.controller object and run some
FileOperations you should get nice set of messages into your
overridden handle\_message function detailing exactly what the module
is doing.  If you further change handle\_message to return false then
it will only tell you what it plans to do, rather than actually doing
it.  You can also tweak based on the message type or message tags (see
flowmonitor.rb for more details) for very fine grained control.  With
this you should be able to put all your logging logic in one spot and
all your dry-run logic in another (you can have multiple observers
that are called in the order of registration until they all pass or
one of them cancels the flow by returning false - so you could
register a logger first and a dry-run observer second).

If you wanted to simple send a message to the observer without
actually wrapping it in an attempt block you can just call
Controller.send\_message with a message object of your own
construction.  The attempt block is just my attempt to provide a very
thin wrapper around what I think is going to be the most common use of
the object.  On the back-end attempt uses only publicly available
methods to do its work, and it could all be done by the library using
direct calls instead.  In the case you asked about...

You can see that this allows a fine-grained approach to logging -
where the logging information actually helps documenting the code
without becoming disturbing. This is where techniques like aspect
oriented programming (AOP) fall short. The granularity of AOP is at
the method level. With our 'flowmonitor' we get control without
sacrificing readability. Thanks to Ruby's closures.


\subsection{Verbose vs. Trace}

The thing to keep in mind are the verbosity (1..3) and trace levels
(1..3). Trace is for debugging purposes, to see where a script
fails. Verbose is for checking activity on the system.

If you would use trace 1 you just want to see major landmarks - like
entering a function.

Trace level 3 is every action and level 2 is somewhere in between.

With verbose level 1 is show changes only - file copied for example.
Level 2 is all actions (check file exists) and level 3 is everything.

So they serve different purposes.
Logging

For Cfruby there are really two levels of logging - outside the
standard 'FATAL, ERROR, WARN, INFO, DEBUG':

1. Display all activity - i.e. check file attributes

2. Display changes only - i.e. display when file attribute changes

So we need 'FATAL, ERROR, WARN, INFO, DEBUG' for program/library
behaviour and additionally 'ACTIVITY,CHANGES' for Cfruby output.

Because this is what the user wants to see when running the engine.
Either he is checking his 'code' or he is tracking changes.

Within these two levels we want to be able to filter on actions. I.e.
a user may want to see changes only on tidy, but all activity on file.
So every action sequence should have it's own logger -
configurable through both the command line and control.

Furthermore, with both levels we also want to be able to distinguish
some levels of severity where the user (typically) can decide on
verbosity. I suggest three levels. The code could look like:

\begin{verbatim}
  logger = Logger.new(FILE)
  logger.addDefault(CHANGES,2)
  logger.addAction('File',ACTIVITY[,1])
\end{verbatim}


And in the code:

\begin{verbatim}
  logger.activity(1,'chmod #{fn} to #{act}')
\end{verbatim}

The logger defaults to no output in the library version of cfruby, but
a user should be able to switch it on, so we can add logging
statements at that level, instead of creating a wrapper.



The flow controller from Cfruby has to support verbose and debug              
(--trace) levels (each 3 levels).                                             
                                                                              
Verbosity is set by the end-user.                                             
                                                                   
\begin{verbatim}
    #   VERBOSE_MINOR    (1)   Minor notification                             
    #   VERBOSE_MAJOR    (2)   Major notification                             
    #   VERBOSE_CRITICAL (3)   Critical notification                          
\end{verbatim}
                                                                              
Debugging is for testing/development:                                          
                
\begin{verbatim}
    #   TRACE_LIBRARY    (1)   Library (scripting) level                      
    #   TRACE_ENGINE     (2)   Engine level                                   
    #   TRACE_ALL        (3)   Trace all                                      
\end{verbatim}
                                                              
                                                                              
Check also ./lib/Cfruby/cfp\_logger.rb                                         
                                                                              
I hope I can retain this behaviour from your library, where I only use        
verbosity levels. A user running at level 3 only sees changes to the          
system:                                                                       
                                                                              
'copied file'                                                                 

'deleted file'                                                                
                                                                              
A user at level 2 also sees activity:                                         
                                                                              
'check file'                                                                  
                                                                              
A user at level 1 sees everything worth mentioning:         
                                                                              
'executing cfcupsd.rb'                                                        
                                                                              
etc.                                                                          
               
                                                                              

